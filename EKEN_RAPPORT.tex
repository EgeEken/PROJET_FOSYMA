\documentclass[a4paper,11pt]{article}
\usepackage[utf8]{inputenc}
\usepackage[T1]{fontenc}
\usepackage{graphicx}
\usepackage{hyperref}
\usepackage{geometry}
\geometry{margin=1in}

% Helper to include figures if available, otherwise show placeholder
\newcommand{\includefigure}[2][]{%
  \IfFileExists{#2}{\includegraphics[#1]{#2}}{\fbox{\textbf{Missing:} #2}}%
}

\begin{document}

%===============================================================================
% COVER PAGE
%===============================================================================
\begin{titlepage}
  \centering
  \vspace*{2cm}
  {\LARGE \textbf{Collaborative Multi-Agent Treasure Retrieval}\\[1ex]
  \large FoSyMa – Project Report 2025}\\[2cm]

  {\Large \textbf{Author:} Tarik Ege Eken (21110611)}\\[1.5cm]

  {\large \textbf{Group \#:} 2}\\[2cm]

  {\large \textbf{Source Code:} \href{https://github.com/EgeEken/PROJET_FOSYMA}{github.com/EgeEken/PROJET_FOSYMA-project}}\\

  \vfill
  {\large \today}
\end{titlepage}

%===============================================================================
% TABLE OF CONTENTS
%===============================================================================
\tableofcontents
\newpage

%===============================================================================
% 1. OVERVIEW
%===============================================================================
\section{Overview}
This document describes the implementation of a decentralized multi-agent system for treasure gathering on an unknown graph. Leveraging the FoSyMa framework atop JADE and GraphStream, a team of agents with distinct roles cooperate to explore, share information, collect resources (gold and diamonds), and return them to a central deposit.

\textbf{Highlights:}
\begin{itemize}
  \item Integrated exploration and resource search via a unified \texttt{SearchBehaviour}.
  \item Periodic full-graph broadcasts (\texttt{ShareMultiMapBehaviour}) and depot-announcements (\texttt{SayHelloBehaviour}).
  \item Resource collectors that compute shortest delivery routes using Dijkstra.
  \item Analytical discussion of algorithmic complexity, communication load, and stopping conditions.
\end{itemize}

%===============================================================================
% 2. SYSTEM ARCHITECTURE
%===============================================================================
\section{System Architecture}
\subsection{JADE Initialization}
The bootstrapper (\texttt{Principal.java}) reads \texttt{agent-custom.json} to configure:
\begin{itemize}
  \item Number and types of agents.
  \item Graph size and generator settings.
  \item Communication radius and resource parameters.
\end{itemize}
Subsequently, it launches agents in dedicated containers matching their roles.

\subsection{UML Diagram}
% Placeholder for UML: insert when available
\begin{figure}[h]
  \centering
  \includefigure[width=0.8\textwidth]{uml_graph.png}
  \caption{Proposed class and package structure.}
\end{figure}

%===============================================================================
% 3. DESIGN DECISIONS
%===============================================================================
\section{Design Decisions}

\subsection{Exploration Strategy}
We implemented Depth-First Search within \texttt{SearchBehaviour}. Agents maintain a stack of frontier nodes:
\begin{enumerate}
  \item \textbf{Sense}: call \texttt{observe()} to gather local adjacency and resource info.
  \item \textbf{Update}: record nodes and edges in \texttt{MapRepresentation}.
  \item \textbf{Traverse}: push unexplored neighbors; pop to continue the search.
\end{enumerate}

\textbf{Merits:} Minimal memory ($O(N)$), straightforward deep exploration.\\
\textbf{Drawbacks:} Not shortest-path optimal; redundant visits possible.\\
\textbf{Complexity:} $O(N+E)$ time, $O(N)$ memory.

\subsection{Information Exchange}
Agents periodically broadcast their entire known map using \texttt{ShareMultiMapBehaviour}. The tanker agent uses \texttt{SayHelloBehaviour} to announce its presence.

\textbf{Merits:} Simple to implement, resilient to message loss.\\
\textbf{Drawbacks:} High bandwidth; no incremental updates.\\
\textbf{Cost:} $O(|V|+|E|)$ bytes per broadcast.

\subsection{Coordination Mechanisms}
Coordination is implicit through shared maps:
\begin{itemize}
  \item Explorers identify unvisited frontiers cooperatively.
  \item Collectors query merged maps to locate nearest resources.
  \item The tanker’s beacon ensures collectors know where to return when full.
\end{itemize}

\textbf{Merits:} Fully decentralized; no leader election.\\
\textbf{Drawbacks:} Latency between updates; potential for redundant effort.

\subsection{Resource Gathering}
Collectors follow a two-phase routine:
\begin{enumerate}
  \item \textbf{Hunting}: random exploration until a resource is detected.
  \item \textbf{Delivery}: compute shortest path to the tanker via Dijkstra and move accordingly.
\end{enumerate}

\textbf{Merits:} Shortest-path guarantee for delivery.\\
\textbf{Drawbacks:} Initial random walk is suboptimal; no resource prioritization.\\
\textbf{Complexity:} Delivery $O(E + V\log V)$ per Dijkstra run.

\subsection{Collision Avoidance}
Agents mark occupied nodes upon observation and avoid moving into them, deferring when necessary. The JADE scheduler resolves simultaneous move requests by delaying one action.

\textbf{Merits:} Prevents deadlocks and state overlap.\\
\textbf{Drawbacks:} Increased latency in congested areas.\\
\textbf{Cost:} $O(d)$ per move decision, where $d$ is node degree.

%===============================================================================
% 4. ALGORITHMIC ANALYSIS
%===============================================================================
\section{Algorithmic Analysis}
\subsection{DFS-Based Exploration}
Described in Section 3.1; stopping when the frontier stack empties.

\subsection{Map Merging}
Upon receiving a peer’s map, agents merge node/edge sets in $O(n_a + n_b + e_a + e_b)$ time. Termination is detected when no new information appears.

\subsection{Shortest-Path Routing}
Collectors use Dijkstra: $O(E + V\log V)$ complexity; optimal under uniform weights; resets backpack once unloading completes.

\subsection{Communication Throughput}
Each tick generates $A$ broadcasts of size $O(|V|+|E|)$, where $A$ is the number of agents. This dominates overall network load.

%===============================================================================
% 5. REFLECTION & FUTURE WORK
%===============================================================================
\section{Reflection \& Future Work}
Our implementation fulfills core exploration and collection requirements, but omits several advanced features:
\begin{itemize}
  \item \textbf{Adversarial Golems}: No modeling of moving threat agents that relocate resources.
  \item \textbf{Collaborative Lockpicking}: Agents act individually rather than pooling capabilities.
  \item \textbf{Incremental Updates}: Full-graph broadcasts could be replaced by delta exchanges.
  \item \textbf{Task Allocation}: Explicit frontier assignment or auction-based role shifts.
\end{itemize}

Enhancements such as dynamic agent roles, event-driven communication, and adversary-aware planning would complete the FoSyMa vision while improving efficiency.

\end{document}
